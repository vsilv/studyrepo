Experiment I \\ \vspace{3 mm} \normalsize \vspace{3 mm}
	\begin{tabular}{| l | r|r|r|r|r|r|r|}
	\hline
        g in m& 0.145 & 0.258 & 0.120 & 0.106 & 0.349 & 0.112 & 0.287 \\ \hline Sg in m& 0.008 & 0.008 & 0.006 & 0.003 & 0.005 & 0.005 & 0.004 \\ \hline
	\end{tabular} \\ \vspace{3 mm} \normalsize \vspace{3 mm}
	\begin{tabular}{| l | r|r|r|r|r|r|r|}
	\hline
        b in m& 0.176 & 0.117 & 0.255 & 0.349 & 0.106 & 0.293 & 0.118 \\ \hline Sb in m& 0.008 & 0.008 & 0.006 & 0.004 & 0.006 & 0.006 & 0.005 \\ \hline
	\end{tabular} \\ \vspace{3 mm} \normalsize \vspace{3 mm}
	\begin{tabular}{| l | r|r|r|r|r|r|r|}
	\hline
        e in m& 0.080 & 0.080 & 0.082 & 0.081 & 0.081 & 0.081 & 0.084 \\ \hline Se in m& 0.001 & 0.003 & 0.002 & 0.002 & 0.003 & 0.002 & 0.002 \\ \hline
	\end{tabular} \\ \vspace{3 mm} \normalsize \vspace{3 mm}
	\begin{tabular}{| l | r|r|r|r|r|r|r|}
	\hline
        f in m& 0.080 & 0.080 & 0.082 & 0.081 & 0.081 & 0.081 & 0.084 \\ \hline Sf in m& 0.001 & 0.003 & 0.002 & 0.002 & 0.003 & 0.002 & 0.002 \\ \hline
	\end{tabular} \\ \vspace{3 mm} \normalsize \vspace{3 mm}
	\begin{tabular}{| l | r|r|r|r|r|r|r|}
	\hline
        beta& 1.214 & 0.453 & 2.125 & 3.292 & 0.304 & 2.616 & 0.411 \\ \hline Sbeta& 0.124 & 0.046 & 0.160 & 0.133 & 0.020 & 0.167 & 0.022 \\ \hline
	\end{tabular} \\ Mittelwert: 0.081\\Standartabweichung: 0.001\\gemessene Brechkraft: $\phi=\frac{1}{f}=12.305 +- 0.176\\ $theoretische Brechkraft: 12.500\\ Experiment II 1. \\ \vspace{3 mm} \normalsize \vspace{3 mm}
	\begin{tabular}{| l | r|r|r|r|r|r|r|}
	\hline
        g in m& 0.111 & 0.176 & 0.090 & 0.160 & 0.093 & 0.187 & 0.086 \\ \hline Sg in m& 0.010 & 0.005 & 0.005 & 0.005 & 0.005 & 0.005 & 0.005 \\ \hline
	\end{tabular} \\ \vspace{3 mm} \normalsize \vspace{3 mm}
	\begin{tabular}{| l | r|r|r|r|r|r|r|}
	\hline
        b in m& 0.114 & 0.079 & 0.165 & 0.085 & 0.152 & 0.078 & 0.179 \\ \hline Sb in m& 0.010 & 0.006 & 0.006 & 0.006 & 0.006 & 0.006 & 0.006 \\ \hline
	\end{tabular} \\ \vspace{3 mm} \normalsize \vspace{3 mm}
	\begin{tabular}{| l | r|r|r|r|r|r|r|}
	\hline
        e in m& 0.056 & 0.055 & 0.058 & 0.056 & 0.058 & 0.055 & 0.058 \\ \hline Se in m& 0.001 & 0.002 & 0.002 & 0.002 & 0.001 & 0.002 & 0.002 \\ \hline
	\end{tabular} \\ \vspace{3 mm} \normalsize \vspace{3 mm}
	\begin{tabular}{| l | r|r|r|r|r|r|r|}
	\hline
        f in m& 0.056 & 0.055 & 0.058 & 0.056 & 0.058 & 0.055 & 0.058 \\ \hline Sf in m& 0.001 & 0.002 & 0.002 & 0.002 & 0.001 & 0.002 & 0.002 \\ \hline
	\end{tabular} \\ \vspace{3 mm} \normalsize \vspace{3 mm}
	\begin{tabular}{| l | r|r|r|r|r|r|r|}
	\hline
        beta& 1.027 & 0.449 & 1.833 & 0.531 & 1.634 & 0.417 & 2.081 \\ \hline Sbeta& 0.185 & 0.044 & 0.164 & 0.051 & 0.147 & 0.041 & 0.186 \\ \hline
	\end{tabular} \\ Mittelwert: 0.056\\Standartabweichung: 0.001\\gemessene Brechkraft: $\phi=\frac{1}{f}=17.706 +- 0.444\\ $theoretische Brechkraft: 17.422\\ Experiment II 2. \\ \vspace{3 mm} \normalsize \vspace{3 mm}
	\begin{tabular}{| l | r|r|r|r|r|}
	\hline
        g in m& 0.107 & 0.168 & 0.082 & 0.081 & 0.182 \\ \hline Sg in m& 0.010 & 0.005 & 0.005 & 0.005 & 0.005 \\ \hline
	\end{tabular} \\ \vspace{3 mm} \normalsize \vspace{3 mm}
	\begin{tabular}{| l | r|r|r|r|r|}
	\hline
        b in m& 0.118 & 0.087 & 0.173 & 0.184 & 0.083 \\ \hline Sb in m& 0.010 & 0.006 & 0.006 & 0.006 & 0.006 \\ \hline
	\end{tabular} \\ \vspace{3 mm} \normalsize \vspace{3 mm}
	\begin{tabular}{| l | r|r|r|r|r|}
	\hline
        e in m& 0.056 & 0.057 & 0.056 & 0.056 & 0.057 \\ \hline Se in m& 0.001 & 0.002 & 0.002 & 0.002 & 0.002 \\ \hline
	\end{tabular} \\ \vspace{3 mm} \normalsize \vspace{3 mm}
	\begin{tabular}{| l | r|r|r|r|r|}
	\hline
        f in m& 0.056 & 0.057 & 0.056 & 0.056 & 0.057 \\ \hline Sf in m& 0.001 & 0.002 & 0.002 & 0.002 & 0.002 \\ \hline
	\end{tabular} \\ \vspace{3 mm} \normalsize \vspace{3 mm}
	\begin{tabular}{| l | r|r|r|r|r|}
	\hline
        beta& 1.103 & 0.518 & 2.110 & 2.272 & 0.456 \\ \hline Sbeta& 0.199 & 0.048 & 0.197 & 0.210 & 0.043 \\ \hline
	\end{tabular} \\ Mittelwert: 0.056\\Standartabweichung: 0.001\\gemessene Brechkraft: $\phi=\frac{1}{f}=17.711 +- 0.193\\ $theoretische Brechkraft: 17.422\\ Experiment III 1.\\ \vspace{3 mm} \normalsize \vspace{3 mm}
	\begin{tabular}{| l | r|r|r|r|r|r|}
	\hline
        b' in m& 0.200 & 0.310 & 0.431 & 0.551 & 0.698 & 0.780 \\ \hline Sb' in m& 0.020 & 0.010 & 0.005 & 0.005 & 0.005 & 0.005 \\ \hline
	\end{tabular} \\ \vspace{3 mm} \normalsize \vspace{3 mm}
	\begin{tabular}{| l | r|r|r|r|r|}
	\hline
        b' in m& 0.818 & 0.735 & 0.590 & 0.467 & 0.346 \\ \hline Sb' in m& 0.005 & 0.005 & 0.005 & 0.005 & 0.005 \\ \hline
	\end{tabular} \\ \vspace{3 mm} \normalsize \vspace{3 mm}
	\begin{tabular}{| l | r|r|r|r|r|r|}
	\hline
        beta& 1.024 & 1.963 & 2.963 & 4.000 & 5.244 & 5.889 \\ \hline Sbeta& 0.049 & 0.074 & 0.074 & 0.074 & 0.074 & 0.074 \\ \hline
	\end{tabular} \\ \vspace{3 mm} \normalsize \vspace{3 mm}
	\begin{tabular}{| l | r|r|r|r|r|}
	\hline
        beta& 5.889 & 5.185 & 4.000 & 2.926 & 1.926 \\ \hline Sbeta& 0.074 & 0.074 & 0.074 & 0.074 & 0.074 \\ \hline
	\end{tabular} \\ 